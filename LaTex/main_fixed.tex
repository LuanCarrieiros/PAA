\documentclass{sbc2023}
\usepackage{float}
\usepackage{graphicx}
% BIBLIOGRAFIA: Removido biblatex - usando natbib do sbc2023.cls
\usepackage[misc,geometry]{ifsym}
\usepackage{fontspec} % Para compilar com XeLaTeX ou LuaLaTeX
\usepackage{fontawesome}
\usepackage{academicons}
\usepackage{color}
\usepackage{hyperref}
\usepackage{aas_macros}
\usepackage[bottom]{footmisc}
\usepackage{supertabular}
\usepackage{afterpage}
\usepackage{url}
\usepackage{pifont}
\usepackage{multicol}
\usepackage{multirow}
\usepackage{subcaption} 
\usepackage{tabularx}
\usepackage{booktabs}

% REMOVIDO: Configuração biblatex (conflitava com natbib)
% Bibliografia será processada com natbib (já incluído no sbc2023.cls)

% Cores personalizadas
\definecolor{orcidlogo}{rgb}{0.37,0.48,0.13}
\definecolor{unilogo}{rgb}{0.16, 0.26, 0.58}
\definecolor{maillogo}{rgb}{0.58, 0.16, 0.26}
\definecolor{darkblue}{rgb}{0.0,0.0,0.0}

% Configuração de hiperlinks
\hypersetup{colorlinks,breaklinks,
            linkcolor=darkblue,urlcolor=darkblue,
            anchorcolor=darkblue,citecolor=darkblue}

% Metadados do Journal
\jid{JBCS}
\jtitle{Journal of the Brazilian Computer Society, 2025, XX:1, }
\doi{10.5753/jbcs.2025.XXXXXX}
\copyrightstatement{This work is licensed under a Creative Commons Attribution 4.0 International License}
\jyear{2025}

\title[Image Similarity Search]{Comparative Performance Analysis of Data Structures for RGB Image Similarity Search: An Empirical Study}

% Autores e Afiliações
\author[Carrieiros et al. 2025]{
\affil{\textbf{Luan Barbosa Rosa Carrieiros}~\href{https://orcid.org/0009-0007-2310-1129}{\textcolor{orcidlogo}{\aiOrcid}}~\textcolor{blue}{\faEnvelopeO}~~[~\textbf{Pontifical Catholic University of Minas Gerais}~|\href{mailto:luan.rosa@sga.pucminas.br}{~\textbf{\textit{luan.rosa@sga.pucminas.br}}}~]}

\affil{\textbf{Diego Moreira Rocha}~\href{https://orcid.org/0000-0002-7110-2026}{\textcolor{orcidlogo}{\aiOrcid}}~~[~\textbf{Pontifical Catholic University of Minas Gerais}~|\href{mailto:diego.moreira@sga.pucminas.br}{~\textbf{\textit{diego.moreira@sga.pucminas.br}}}~]}

\affil{\textbf{Iago Fereguetti Ribeiro}~\href{https://orcid.org/0000-0003-3052-3016}{\textcolor{orcidlogo}{\aiOrcid}}~~[~\textbf{Pontifical Catholic University of Minas Gerais}~|\href{mailto:iago.fereguetti@sga.pucminas.br}{~\textbf{\textit{iago.fereguetti@sga.pucminas.br}}}~]}

\affil{\textbf{Bernardo Ferreira Temponi}~\href{https://orcid.org/0000-0001-4892-5537}{\textcolor{orcidlogo}{\aiOrcid}}~~[~\textbf{Pontifical Catholic University of Minas Gerais}~|\href{mailto:bernardo.temponi@sga.pucminas.br}{~\textbf{\textit{bernardo.temponi@sga.pucminas.br}}}~]}
}

\begin{document}

\begin{frontmatter}
\maketitle

\begin{mail}
PUC Minas, Instituto de Ciências Exatas e Informática (ICEI), Av. Dom José Gaspar, 500, Coração Eucarístico, Belo Horizonte, MG, 30535-901, Brazil.
\end{mail}

\begin{abstract}
\textbf{Abstract.} \\
Image similarity search in RGB color space represents a fundamental challenge in computer vision and database systems, requiring efficient data structures to balance search performance with precision requirements. This paper presents a comprehensive empirical analysis of five distinct data structures for RGB image similarity search: \textbf{(i)} Linear Search (brute force baseline), \textbf{(ii)} Hash Search (3D spatial grid hashing), \textbf{(iii)} Hash Dynamic Search (adaptive expansion spatial hashing), \textbf{(iv)} Octree Search (3D recursive spatial tree), and \textbf{(v)} Quadtree Search (2D spatial tree projection). The algorithms were implemented in C++17 with compiler optimizations to ensure fair performance comparison. Extensive experiments were conducted on both synthetic datasets (100 to 50 million images) and real image collections (7,721 natural images across 8 categories) using Euclidean distance in RGB space with a similarity threshold of 50.0. Results reveal that Hash Search achieves superior search performance (18ms for 50M images) while Linear Search dominates insertion operations. Unexpectedly, 2D spatial structures (Quadtree) consistently outperformed 3D equivalents (Octree), and hash-based methods demonstrated a precision-speed trade-off, finding 85-86\% of similar images compared to brute force methods. The study provides quantitative evidence for practical algorithm selection in image similarity applications and establishes performance benchmarks for RGB similarity search systems.
\end{abstract}

\begin{keywords}
Image similarity search; Data structures; Hash tables; Spatial indexing; Octree; Quadtree; RGB color space; Performance analysis; Empirical study; C++.
\end{keywords}

\end{frontmatter}

\section{Introduction}
\label{sec:intro}

Image similarity search in high-dimensional color spaces is a cornerstone problem in computer vision, content-based image retrieval, and multimedia database systems. The challenge lies in efficiently indexing and querying large collections of images based on perceptual similarity, typically measured using distance metrics in RGB color space. As image datasets grow exponentially in size—from thousands to millions of images—the choice of underlying data structure becomes critical for system performance and scalability.

Traditional approaches range from brute force linear search, which guarantees complete recall but suffers from linear time complexity, to sophisticated spatial indexing techniques that exploit the geometric properties of color space. Hash-based methods offer promising constant-time access patterns, while tree structures provide logarithmic search complexity with spatial locality advantages. However, the practical performance of these approaches in real-world scenarios often deviates significantly from theoretical predictions due to implementation details, memory hierarchy effects, and data distribution characteristics.

In this study, we present a comprehensive empirical analysis of five fundamental data structures for RGB image similarity search: Linear Search, Hash Search, Hash Dynamic Search, Octree Search, and Quadtree Search. Each algorithm was implemented in C++17 with careful attention to performance optimization and fair comparison methodology. Our experimental evaluation encompasses both controlled synthetic datasets (ranging from 100 to 50 million images) and real-world image collections, providing insights into the practical trade-offs between search speed, insertion performance, memory consumption, and precision.

\section{Theoretical Background}
\label{sec:background}

Image similarity search in RGB color space can be formalized as a nearest neighbor problem in a three-dimensional Euclidean space. Given a query point $q = (r_q, g_q, b_q)$ and a similarity threshold $\tau$, the objective is to efficiently retrieve all images $I_i = (r_i, g_i, b_i)$ such that the Euclidean distance $d(q, I_i) \leq \tau$, where:

\begin{equation}
d(q, I_i) = \sqrt{(r_q - r_i)^2 + (g_q - g_i)^2 + (b_q - b_i)^2}
\label{eq:euclidean_distance}
\end{equation}

This section presents the theoretical foundations and complexity analysis of each data structure implemented in our comparative study.

\subsection{Linear Search (Brute Force)}

Linear Search represents the baseline approach, examining every image in the dataset sequentially. Despite its simplicity, it guarantees 100\% recall and serves as the ground truth for precision evaluation.

\textbf{Time Complexity:}
\begin{itemize}
    \item Insertion: $O(1)$ - direct array append
    \item Search: $O(n)$ - sequential scan of all elements
    \item Space: $O(n)$ - stores only the image data
\end{itemize}

% Continuar com outras seções...
% [CONTENT TRUNCATED FOR SPACE - incluir resto das seções aqui]

\section{Conclusion and Future Work}
\label{sec:conclusion}

This comprehensive empirical study provides quantitative evidence for practical data structure selection in RGB image similarity search applications. Our evaluation of five distinct approaches—Linear Search, Hash Search, Hash Dynamic Search, Octree Search, and Quadtree Search—reveals that Hash-based methods achieve superior search performance while maintaining acceptable precision trade-offs.

Key contributions include: (i) comprehensive performance benchmarks across synthetic and real datasets, (ii) quantification of the precision-speed trade-off in spatial indexing methods, (iii) demonstration of the 2D vs. 3D performance paradox in tree structures, and (iv) validation of the critical importance of memory management for large-scale evaluation.

\section*{Code Availability}
\label{sec:code_availability}

The complete source code developed for this study, including implementations of all five data structures, comprehensive benchmarking framework, and experimental evaluation tools, is publicly available for reproducibility and further research.

Repository: \url{https://github.com/LuanCarrieiros/PAA}

% BIBLIOGRAFIA COM NATBIB (compatível com sbc2023.cls)
\bibliographystyle{sbc}
\bibliography{references}

\end{document}